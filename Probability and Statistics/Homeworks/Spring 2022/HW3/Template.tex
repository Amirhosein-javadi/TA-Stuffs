%!TeX program = xelatex
\documentclass[a4paper]{article}
\usepackage{graphicx}
\usepackage{amsmath, amsfonts, geometry, float, listings, enumerate, multicol}
\usepackage{multicol, float, color, colortbl}
\usepackage{tikz, titlesec, parskip, pgfplots, filecontents}
\usepackage{hyperref}
\usepackage{amsmath}
\usepackage{bidipoem}
\usepackage{tikz, titlesec, parskip}
\usepackage{tikz,pgfplots}
\usepackage[americanvoltages,fulldiodes,siunitx]{circuitikz}
\usetikzlibrary{shapes,arrows}
\usetikzlibrary{angles,quotes}
\usepackage{enumitem}

\titlespacing{\section}{0pt}{10pt}{0pt}
\titlespacing{\subsection}{0pt}{10pt}{0pt}
\titlespacing{\subsubsection}{0pt}{10pt}{0pt}

\usetikzlibrary{calc,patterns,through}
\newcommand{\arcangle}{%
	\mathord{<\mspace{-9mu}\mathrel{)}\mspace{2mu}}%
}

\renewcommand{\baselinestretch}{1.4}
 \geometry{
 a4paper,
 total={170mm,257mm},
 left=20mm,
 top=20mm,
 }
\usepackage{fancyhdr}
\usepackage{indentfirst}
\pagestyle{fancy}
\fancyhf{}
\rhead{\textbf{آمار و احتمال مهندسی}}
\lhead{\textbf{تمرین عملی سری سوم}}
\cfoot{(\space \space \space \space \textbf{\thepage}  \space \space \space)}
\renewcommand{\headrulewidth}{1pt}
\renewcommand{\footrulewidth}{1pt}

 
\usepackage{xepersian}
\setlatintextfont{Times New Roman}
\settextfont{XB Niloofar}
\DefaultMathsDigits

\makeatletter
\bidi@patchcmd{\@Abjad}{آ}{الف}
{\typeout{Succeeded in changing آ into الف}}
{\typeout{Failed in changing آ into الف}}
\makeatother
\PersianAlphs

\begin{document}
\begin{minipage}{0.6\textwidth}
\begin{bf}
\begin{center}
	به نام خدا\\
	\vspace{0.25cm}
	دانشگاه صنعتی شریف\\
	\vspace{0.25cm}
	دانشکده مهندسی برق\\
	\vspace{0.5cm}

\large
گروه دکتر کرباسی - آمار و احتمال مهندسی \\
نیم سال دوم
۱۴۰۱-۱۴۰۰\\
\Large
\vspace{0.4cm}
تمرین عملی سری سوم \\
\end{center}
\end{bf}
\normalsize
\end{minipage} \hfill
\begin{minipage}{0.35\textwidth}
\begin{flushleft}
\includegraphics[width=0.6\textwidth]{Shariflogo.png}\\ \large
\end{flushleft}

 \end{minipage}
\\

\rule[0.1\baselineskip]{\textwidth}{1.5pt}

\large

\section*{
لطفاً به نکات زیر توجه بفرمایید: (رعایت نکردن این موارد باعث کاهش نمره می‌شود.)
}
\begin{enumerate}
	\item 
نتایج و پاسخ های خود را در یک فایل با فرمت zip به نام
\lr{HW2\_StudentID\_Name}
 در سایت  
\href{https://quera.org/overview/add_to_course/course/10631}{\lr{Quera}} 
 قرار دهید. همچنین فایل پایتون خود را به همان نام در قسمت مخصوص به خود آپلود کنید.
	\item 
کسب نمره کامل در هر سؤال مستلزم تحویل  \textbf{کدها} و \textbf{توضیحات} می‌باشد. 
\item 
برای سؤالات، باید روشی که استفاده کرده‌اید را توضیح  و نتایجی که گرفته‌اید را ارائه دهید. این توضیحات می‌تواند در یک فایل  .pdf  و یا در یک فایل  .ipynb باشد. 
\item
فایل های تحویلی شما دو بخش میباشند، یک بخش فایل .zip که شامل فایل .ipynb کد و گزارش شما میباشد، یک بخش هم کد های هر سوال به شکل جداگانه میباشند که باید در فرمت .py در سامانه کوئرا در کنار فایل .zip آپلود شوند.(برای مثال اگر تمرین شامل 3 سوال بود، باید علاوه بر فایل .zip که تحویل مصحح میشود، 3 فایل .py در سامانه کوئرا در محل بارگذاری مشخص شده آپلود کنید.)
\item 
کدهای خود را خوانا بنویسید و کامنت‌‌گذاری کنید. در plot های خود عنوان، label و خط‌کشی‌های مناسب را اضافه کنید.
\item
در طول ترم امکان ارسال با تاخیر پاسخ  همه‌ی تمارین تا سقف پنج روز و در مجموع دوازده روز وجود دارد. پس از گذشت این مدت، پاسخ‌های ارسال‌شده پذیرفته نخواهند بود. همچنین، به ازای هر روز تأخیر غیر مجاز  بیست درصد از نمره تمرین به صورت ساعتی کسر خواهد شد.
\item
کدهای شما تماماً باید توسط خودتان نوشته شده باشند. هرگونه استفاده از کد دیگران به هر شکل ممکن، تقلب محسوب می‌شود و نمره تمرین کامپیوتری جاری صفر خواهد شد. پس در هیچ صورت کدهای خود را برای دیگران ارسال نکنید.
\item 
ابهام يا اشكالات خود را مي توانيد  از طریق
\href{mailto:smmzdr@gmail.com}{\LR{smmzdr@gmail.com}}
یا 
\href{mailto:javadiamirhosein.2000@gmail.com}{\LR{javadiamirhosein.$2000$@gmail.com}}
مطرح نماييد.
\item 
مهلت تحویل:  
\textbf{نیمه شب جمعه 26 فروردین}
\end{enumerate}
\rule[0.1\baselineskip]{\textwidth}{1.5pt}

\clearpage
\section{‌تلفنچی نمایی}
فرض کنید شما در مرکز مخابرات نشسته‌اید و به زمان‌هایی که یک تماس تلفنی برقرار می‌شود نگاه  می‌کنید. با بررسی دقیق‌تر این تماس‌ها می‌بینید که فاصله‌ی بین زمان‌هایی که  یک تماس برقرار می‌شود از توزیع نمایی با پارامتر $ 0.3 $ پیروی می‌کند و این فاصله‌ها بین تماس‌های مختلف مستقل از هم هستند. چون لحظه‌ی برقراری تماس‌ها تصادفی است و شما از توزیع که روی فاصله ی زمانی بین تماس‌ها وجود دارد آگاهید، سوالی برای شما پیش می‌آید و آن توزیع حاکم بر تعداد تماس‌ها در واحد زمان است. برای این منظور شما به یک شبیه سازی روی می‌آوردید. از متغیر تصادفی
$ X \sim \text{Exponential}(0.3) $
$ 1000 $ نمونه می‌گیرید و فرض می‌کنید این‌ها فاصله‌ی بین تماس‌های مختلف هستند. حال به بازه‌های زمان به طول  1 نگاه می‌کنید و تعداد تماس‌های گرفته شده در هر بازه ی زمانی به طول واحد را یادداشت می‌کنید. یعنی به باز ه های
$ [0,1] $,
$ [1,2] $
تا انتها (تا جایی که تماسی گرفته شده باشد)
نگاه کرده و در هر
بازه تعداد تماس‌ها را خواهید شمرد. سپس هیستوگرام نرمالیزه‌شده تعداد تماس‌ها در بازه‌های به طول واحد را رسم می‌کنید. انتظار دارید این هیستوگرام متناظر با تابع جرم احتمال چه متغیر تصادفی با چه پارامترهایی باشد؟ تابع جرم
احتمال آن متغیر تصادفی را همراه این هیستوگرام بر روی یک 
\lr{plot}
ترسیم کنید.

توابع پیشنهادی:
\lr{np.random.exponential}
از کتاب‌خانه‌ی 
\lr{numpy}
\section{جمع ماکسیمم ها}
متغیر تصادفی زیر را در نظر بگیرید:
\begin{gather*}
	X_i \sim Bernoulli(p = 0.4)
\end{gather*}
متغیر های تصادفی $Z, Y_1, Y_2, Y_3$ را به شکل زیر تعریف میکنیم:
\begin{gather*}
	Y_1 = max(X_1, X_2)
	\\
	Y_2 = max(X_2, X_3)
	\\
	Y_3 = max(X_3, X_1)
	\\
	Z = Y_1 + Y_2 + Y_3
\end{gather*}
به ازای 
$ n = 1 : 10000 $
از $X_1, X_2, X_3$ $n$ نمونه بگیرید و $Z, Y_1, Y_2, Y_3$ را تشکیل دهید.
\begin{enumerate}
	\item
	امیدریاضی و واریانس $Z$ به ازای $n$ های مختلف را در دو آرایه \lr{E} و \lr{Var} ذخیره کنید. 
	\item
	آرایه \lr{Cml\_avg\_E} را تعریف کرده و میانگین k عضو ابتدایی \lr{E} را در عضو \lr{k}ام آن ذخیره کنید.
	\item
	آرایه \lr{Cml\_avg\_Var} را تعریف کرده و میانگین k عضو ابتدایی \lr{Var} را در عضو \lr{k}ام آن ذخیره کنید.
	\item
	نمودار \lr{Cml\_avg\_E} و \lr{Cml\_avg\_Var} رابر حسب $n$ رسم کنید. آیا با افزایش $n$ واریانس و امیدریاضی تجربی به مقدار خاصی میل میکنند؟ این مقادیر را بصورت تئوری نیز حساب کنید و با نتیجه بدست آمده مقایسه کنید.
	\item
	آرایه های \lr{Var\_Y1, Var\_Y2, Var\_Y3} را تعریف کنید و واریانس های $Y_1, Y_2, Y_3$ را به ازای $n$ های مختلف در آنها ذخیره کنید.
	\item
	آرایه \lr{Var\_Y\_Sum} را جمع سه آرایه ای که در قسمت قبل تعریف کردید قرار دهید.
	\item
	آرایه \lr{Cml\_avg\_Y\_Sum } را تعریف کرده و میانگین k عضو ابتدایی \lr{Var\_Y\_Sum} را در عضو \lr{k}ام آن ذخیره کنید.
	\item
	نمودار \lr{Cml\_avg\_Y\_Sum} را بر حسب $n$ رسم کنید. آیا با افزایش $n$ جمع واریانس های تجربی $Y_1, Y_2, Y_3$ به مقدار خاصی میل میکند؟ اگر بله این مقدار را با مقداری که در قسمت 4 بدست آوردید مقایسه کرده و در صورت متفاوت بودن آن را توجیه کنید.
\end{enumerate}
\section{
	جمع رئوس (امتیازی)
}
گراف زیر را در نظر بگیرید. هر کدام از رئوس این گراف یک عدد مخصوص دارد و هدف ما پیدا کردن مجموع عددهای رئوس این گراف  در هر راس از گراف است. در اینجا هر راس می‌تواند با راس همسایه تبادل اطلاعات کند ولی رئوس حافظه محدودی دارند و می‌توانند فقط تعداد محدودی عدد در خود ذخیره کنند. یال‌ها به دلخواه شماره‌گذاری شده‌اند و عدد هر راس داخل آن در پرانتز نوشته شده‌است. 
\begin{center}
	
	\begin{tikzpicture}[main/.style = {draw, circle, minimum width=15mm, minimum height=15mm}] 
		\node (1) [main]  at (1.5,6) {$A(1.6)$};
		\node (2) [main]  at (6.5,6) {$B(0.8)$};
		\node (3) [main]  at (9,3) {$C(1.6)$};
		\node (5) [main]  at (1.5,0) {$E(1.4)$};
		\node (4) [main]  at (6.5,0) {$D(0.5)$};
		\node (6) [main]  at (-1,3) {$F(3.2)$};
		\draw[-]  (1) -- (2) node[midway,above] {$1$};
		\draw[-]  (2) -- (3) node[midway,above] {$2$};
		\draw[-]  (3) -- (4) node[midway,right] {$3$};
		\draw[-]  (4) -- (5) node[midway,below] {$4$};
		\draw[-]  (5) -- (6) node[midway,left] {$5$};
		\draw[-]  (6) -- (1) node[midway,left] {$6$};
		\draw[-]  (3) -- (6) node[midway,below] {$7$};
	\end{tikzpicture} 
\end{center}
یک ایده‌ برای تخمین مجموع مقادیر رئوس این است که هر راس نمونه‌ی تصادفی از متغیر تصادفی نمایی با  با نرخ عدد آن راس  تولید کند و نمونه‌ی خود را با راس همسایه تبادل ‌کند و پس از آن دو راس همسایه نمونه‌ی خود را با مینیمم نمونه‌ی خود و همسایه جایگزین ‌کنند. پس از چند مرحله همه رئوس مینیمم نمونه‌های تصادفی گره‌های مختلف را میبینند و الگوریتم همگرا می‌شوند و گره‌ها میتوانند مقدار مینیمم را ذخیره ‌کنند. حال می‌دانیم مینیمم $ n $ متغیر تصادفی نمایی با نرخ‌های 
$ \lambda_{i} $،
یک متغیر تصادفی نمایی با نرخ 
$ \sum_{i=1}^{n} \lambda_{i} $ 
است. پس اگر تعداد مناسبی این الگوریتم را تکرار کنیم، همه رئوس به چند نمونه
\lr{iid}
از متغیر نمایی با پارامتر مجموع مطلوب ما دسترسی دارند و می‌توانند با تخمین 
\lr{Maximum Likelihood}
پارامتر نرخ متغیر نمایی را تخمین بزنند.
\begin{enumerate}
	\item 
	می‌توانید یال‌ها را شماره‌گذاری کنید یا از شماره‌گذاری پیشنهادی استفاده کنید. در هر راند به ترتیب‌، یالی را انتخاب کنید و بین دو نمونه‌ی رئوس دو سر یال مینیمم‌گیری کنید و در صورت نیاز نمونه‌ی رئوس را به روز رسانی کنید. 
	\item 
	حداقل تعداد راند مورد نیاز برای این که مطمئن شویم نمونه‌ی مینیمم به همه‌ی گره‌ها رسیده است چه قدر است؟ 
	\item 
	الگوریتم بالا را اجرا کنید و مجموع اعداد رئوس را با حداکثر خطای $ 0.1 $ به دست بیاورید.
\end{enumerate}
\end{document}