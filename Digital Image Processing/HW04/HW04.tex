\documentclass[a4paper]{article}
\usepackage{graphicx}
\usepackage{amsmath, amsfonts, geometry, float, listings, enumerate, multicol}
\usepackage{multicol, float, color, colortbl}
\usepackage{tikz, titlesec, parskip, pgfplots, filecontents}
\usepackage{hyperref}
\usepackage{amsmath}
\usepackage{tikz, titlesec, parskip}
\usepackage{tikz,pgfplots}
\usepackage[americanvoltages,fulldiodes,siunitx]{circuitikz}
\usetikzlibrary{shapes,arrows}
\usepackage{enumitem}
\titleformat*{\subsubsection}{\LARGE\bfseries}
\usepackage{subcaption}
\usepackage{caption}
\titlespacing{\section}{0pt}{10pt}{0pt}
\titlespacing{\subsection}{0pt}{10pt}{0pt}
\titlespacing{\subsubsection}{0pt}{10pt}{0pt}



\usetikzlibrary{calc,patterns,through}
\newcommand{\arcangle}{%
	\mathord{<\mspace{-9mu}\mathrel{)}\mspace{2mu}}%
}

\renewcommand{\baselinestretch}{1.4}
 \geometry{
 a4paper,
 total={170mm,257mm},
 left=20mm,
 top=20mm,
 }
\usepackage{fancyhdr}
\usepackage{indentfirst}
\pagestyle{fancy}
\fancyhf{}
\rhead{\textbf{پردازش تصاویر دیجیتال}}
\lhead{\textbf{تمرین سری چهارم}}
\cfoot{(\space \space \space \space \textbf{\thepage}  \space \space \space)}
\renewcommand{\headrulewidth}{1pt}
\renewcommand{\footrulewidth}{1pt}

 
\usepackage{xepersian}
\setlatintextfont{Times New Roman}
\settextfont{XB Niloofar}
\setdigitfont{XB Niloofar}
\DefaultMathsDigits

\makeatletter
\bidi@patchcmd{\@Abjad}{آ}{الف}
{\typeout{Succeeded in changing آ into الف}}
{\typeout{Failed in changing آ into الف}}
\makeatother
\PersianAlphs

\begin{document}
\begin{minipage}{0.6\textwidth}
\begin{bf}
\begin{center}
	به نام خدا\\
	\vspace{0.25cm}
	دانشگاه صنعتی شریف\\
	\vspace{0.25cm}
	دانشکده مهندسی برق\\
	\vspace{0.5cm}

\large
دکتر عمادالدین فاطمی‌زاده - پردازش تصاویر دیجیتال \\
نیم سال دوم
۱۴۰۱-۱۴۰۰\\
\Large
\vspace{0.4cm}
تمرین عملی سری دوم\\
\end{center}
\end{bf}
\normalsize
\end{minipage} \hfill
\begin{minipage}{0.35\textwidth}
\begin{flushleft}
\includegraphics[width=0.5\textwidth]{Shariflogo.png}\\ \large
\end{flushleft}

 \end{minipage}
\\

\rule[0.1\baselineskip]{\textwidth}{1pt}

\large
\section*{
لطفاً به نکات زیر توجه بفرمایید: (رعایت نکردن این موارد باعث کاهش نمره می‌شود.)
}

\begin{enumerate}
	\item 
نتایج و پاسخ های خود را در یک فایل با فرمت zip به نام
\LR{HW$2$-Name-StudentNumber}
 در سایت  
\href{https://quera.org/course/add_to_course/course/10598/}{\lr{Quera}} 
 قرار دهید. همچنین فایل پایتون یا متلب خود را به همان نام در قسمت مخصوص به خود آپلود کنید.
\item 
کسب نمره کامل در هر سوال مستلزم تحویل  
\textbf{کدها (40 نمره)}
 و
\textbf{توضیحات (30 نمره)}
و
\textbf{نتایج (30 نمره)}
 می‌باشد. 
\item 
کدهای شما تماماً باید توسط خودتان نوشته شده باشند. هرگونه استفاده از کد دیگران، اعم از دوستان و اینترنت، به هر شکل ممکن، تقلب محسوب می‌شود و نمره تمام تمرینات جاری و تمام تمرینات قبلی صفر خواهد شد. با اجرای این کدها باید همان نتایجی که فرستاده اید قابل بازیابی باشند. برنامه شما باید به گونه‌ای باشد که بدون نیاز به هیچ تغییری قابل اجرا باشد، در غیر این‌ صوررت هیچ نمره‌ای تعلق نخواهد گرفت. 
\item 
برای تمام سؤالات، باید جزئیات روشی که استفاده کرده‌اید را توضیح دهید و نتایجی که گرفته‌اید را ارائه دهید. این توضیحات می‌تواند در یک فایل  pdf  و یا در یک فایل  ipynb باشد. در توضیحات، باید اشاره کامل به کارهایی که انجام داده‌اید بنمایید به طوری که یک شخص آگاه از موارد درس بتواند به آسانی متوجه کاری که شما انجام داده‌اید شود.
\item 
در طول ترم امکان ارسال با تاخیر پاسخ  همه‌ی تمارین تا سقف شش روز و در مجموع بیست و یک روز وجود دارد. پس از گذشت این مدت، پاسخ‌های ارسال‌شده پذیرفته نخواهند بود. همچنین، به ازای هر روز تأخیر غیر مجاز  بیست درصد از نمره تمرین به صورت ساعتی کسر خواهد شد.
\item 
 اگر از
\lr{Jupyter notebook} 
 استفاده می‌کنید، میتوانید خروجی‌ها‌ را پاک کنید تا حجم فایل تحویلی زیاد نشود.
\item 
مهلت تحویل: 19 فروردین ساعت 23:59 
\item 
نام طراح هر سوال در زیر آن نوشته شده است و شما میتوانید سوالات خود را از طریق ایمیل یا تلگرام از طراح سوال بپرسید.

\end{enumerate}
\rule[0.1\baselineskip]{\textwidth}{1pt}

\clearpage
\section{فیلتر ماسک چرخان}
\textbf{طراح :‌ }
\vspace{0.5cm}
\\
در این مسئله می‌خواهیم عملکرد روش فیلتر ماسک چرخان را در کاهش نویز بررسی کنیم. استفاده از این روش در نرم کردن تصویر می‌تواند از کِدِرشدگی لبه‌ها جلوگیری کند. نحوه‌ی انجام این روش به این گونه است که حول پیکسل $ P_{x} $ یک پنجره به اندازه $ w \times w $ در نظر می‌گیریم. در داخل این پنجره ماسکی به اندازه $ m \times m $  که $ n < w $ ‌است را حرکت داده و میانگین و واریانس نقاط داخل ماسک را حساب میکنیم.
مقدار نهایی در پیکسل $ P_{x} $  برابر با مقدار میانگین ماسکی است که به ازای آن ماسک مقدار واریانس کمینه گردیده است. تابع ای بنویسید که تصویر ورودی، اندازه پنجره $ w $ و اندازه ماسک $ m $ را گرفته و تصویر خروجی فیلتر شده را تحویل دهد.
‌تصاویر 
\lr{P1.png}
 و
\lr{P1\_noisy.png}
 را بارگذاری کنید و این روش را به ازای  و
 $ w = 5 $
 و 
$   m = 3 $
 بر روی آنها اعمال کرده و خروجی‌ها را ترسیم کنید. همچنین، فیلترِ ساده‌ی میانگین
 $  5 \times 5  $
 را نیز روی تصاویر اعمال کنید. تفاوت بین نتایج را ذکر کرده و دلایل آن‌ها را توضیح دهید.


\section{فیلترینگ تکراری}
\textbf{طراح : }
\vspace{0.5cm}
\\
در این مسئله می‌خواهیم عملکرد روش  فیلترینگ تکراری را در کاهش کِدِرشدگی تصویر بررسی کنیم. خروجی این روش به صورت تکراری به سمت تصویری میل می‌کند که عمل فیلتر معکوس بر روی آن انجام شده است. در هر تکرار عملیات زیر انجام می‌شود:
\begin{gather} 
	\hat{f}_{n+1}(x,y) = P \Big[ \hat{f}_{n}(x,y) + \beta \big(g(x,y) - (\hat{f}_{n}(x,y) * d(x,y))\big)\Big]
	\\
	P [\hat{f}_{n}(x,y)] = \begin{cases}
		\hat{f}_{n}(x,y)  & \hat{f}_{n}(x,y) \geq 0
		\\ 
		0 & \hat{f}_{n}(x,y) < 0
	\end{cases}
\end{gather}
که در آن 
$ g(x,y) $
تصویر داده شده،
$ d(x,y) $
فیلتر محوشدگی است و 
$ \beta $
نیز عدد مثبتی است که سرعت همگرایی الگوریتم را مشخص میکند. منظور از نماد $ * $ نیز عمل کانولوشن دو بعدی است. تصویر
 \lr{P2.tif}
 را بارگذاری کنید.
\begin{enumerate}
	\item 
	به ازای $ \beta $ مناسب و تکرارهای مختلف عملکرد الگوریتم فیلترینگ تکراری را در حذف کِدِرشدگی تصویر بررسی کنید. افزایش تعداد تکرارها باعث میشود که تصویر  واضح‌تر گردد اما از طرفی هم اگر تعداد تکرارها را خیلی افزایش دهیم، اعوجاج تصویر بیشتر میشود. به ازای $ 3  $ تکرار مختلف این رفتار الگوریتم را در سه تصویر نشان دهید. در حالت اول تصویر همچنان کِدِر هست ولی اعوجاج ندارد، در حالت دوم تصویر بیشترین وضوح را دارد و نیز دچار اعوجاج نشده و در حالت سوم تصویر دچار اعوجاج شده‌است. با ذکر دلیل مشاهدات خود را توجیه کنید. مقدار $ \beta $‌ و تعداد تکرارهای هر سه حالت را گزارش کنید.
	\item 
روش
\lr{ inverse filter }
معرفی شده در بخش $ 5.7 $ کتاب درسی را به ازای شعاع‌های قطع مختلف برای این تصویر پیاده‌سازی کنید و نتایج این روش را با روش فیلترینگ تکراری مقایسه کنید.
\\
در روش 
\lr{ inverse filter }
از
 $ H(u,v)=\mathcal{F}{d(x,y)} $ 
استفاده کنید.
\end{enumerate}

\section{فیلتر وینر}
\textbf{طراح : }
\vspace{0.5cm}
\\
در این مسئله می‌خواهیم عملکرد فیلتر وینر را روی تصویر
\lr{P3.jpg}
 بررسی کنیم:
\begin{enumerate}
	\item 
	\lr{DFT}
	 دو بعدیِ سیگنال اولیه
	  $ F(u,v) $
	   را حساب کنید.
	\item 
	طیف توان تصویر 
	$ S_{f}(u,v) $
	 را حساب کنید.
	\item 
	ویز گوسی سفید با
	 $ \sigma = 10 $ 
	 را به تصویر اضافه کرده و آن را \lr{corrupted} بنامید.
	\item 
	تبدیل فیلتر بهینه وینر را توسط فرمول زیر حساب کنید:
	\begin{equation}
		\frac{H^{*}(u,v)}{|H^{*}(u,v)|^{2}+\frac{S_{n}(u,v)}{S_{f}(u,v)}}
	\end{equation}
	که در آن 
	$ H(u,v) $
	را می‌توانیم 1 گرفته و 
	$ S_{n} $
	را نیز 
	$ N\sigma^{2} $ 
	بگیریم. مقدار مناسب $ N $ را با ذکر دلیل تعیین کنید.
	\item 
	با ضرب تبدیل فوریه دو بعدی تصویر \lr{corrupted} در تبدیل فیلتر بهینه‌ی وینر تخمین $ F(u,v) $ را بدست آورید.
	\item
	با گرفتن عکس تبدیل فوریه‌ی دو بعدی از $ F(u,v) $ تخمین سیگنال اولیه $ f(x,y) $ را بدست آورید.
	\item 
	خروجی هر قسمت را نشان داده و تابع \lr{point spread function} متناظر با تبدیل فیلتر وینر در قسمت 4 را نیز نمایش دهید.
\end{enumerate}

\section{تقسیم‌بندی تصویر بر اساس رنگ}
\textbf{طراح : }
\vspace{0.5cm}
\\
تصویر \lr{P5.png} را بارگذاری کنید. در این تصویر کلاه‌های با رنگ‌های مختلف وجود دارد که می‌خواهیم با استفاده از رنگ هر کلاه آن را جدا کنیم.
\begin{enumerate}
	\item 
با استفاده از روش تقسیم‌بندی در فضای \lr{HSI} هر یک از پنج کلاه را از تصویر استخراج کنید. مراحل انجام روش را به دقت توضیح دهید. برای راهنمایی میتوانید به بخش $ 6.7.1  $کتاب مراجعه کنید.
	\item 
این بار هر یک از پنج کلاه را با استفاده از تقسیم‌بندی در فضای \lr{RGB} از تصویر استخراج کنید. مراحل انجام روش را به دقت توضیح دهید. برای راهنمایی میتوانید به بخش $ 6.7.2 $ کتاب مراجعه کنید.
	\item 
	عملکرد الگوریتم تقسیم‌بندی را در دو فضا با یکدیگر مقایسه کنید.
\end{enumerate}

\section{بهبود کیفیت تصویر رنگی با استفاده از هیستوگرام}
\textbf{طراح : }
\vspace{0.5cm}
\\
تصاویر 
\lr{P6\_1.tif}،  
\lr{P6\_2.tif}  
و 
\lr{P6\_3.tif} 
را بارگذاری کنید.
\begin{enumerate}
	\item 
در هر تصویر بر روی کانال‌های 
\lr{R}،
\lr{G}
و
\lr{B} 
عمل یکسان‌سازی هیستوگرام را انجام دهید. سپس، سه کانال را ادغام کرده و خروجی رنگی را تولید کنید. تصاویر بهبود یافته از این طریق را در کنار تصاویر اولیه نمایش داده و تغییرات به‌وجود آمده در هر تصویر را توجیه کنید.
	\item 
در تصویر
\lr{P6\_3.tif}
 با استفاده از هیستوگرام هر یک از کانال‌ها یک هیستوگرام متوسط تولید کنید و براساس این هیستوگرام متوسط یک تابع یکسان‌ساز هیستوگرام بدست آورید. این تابع را به هر یک از کانال‌های
\lr{R}،
\lr{G}
و
\lr{B} 
 اعمال کرده و تصویر رنگی ترکیب شده از این کانال‌ها را نمایش دهید. این تصویر را با تصویر بدست آمده از قسمت 1 برای
\lr{P6\_3.tif}
  مقایسه کنید و تفاوت‌های آن‌ها را توضیح دهید.
  \item 
تصویر
\lr{ P6\_3.tif}
 را به فضاهای 
 \lr{YUV}، 
 \lr{YIQ}
 و
 \lr{YCbCr }
 ببرید. قسمت 1 را در هر یک از این فضاها تکرار کرده و پس از عمل یکسان‌سازی هیستوگرام در هر کانال، تصویر رنگی ترکیب شده از این کانال‌ها را نمایش دهید. این تصاویر را با تصویر بدست آمده از قسمت 1 برای
\lr{P6\_3.tif}
 مقایسه کنید و تفاوت‌های آن‌ها را توضیح دهید.
  
\end{enumerate}


\end{document}
